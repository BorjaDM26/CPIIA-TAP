\chapter{Entorno Tecnol\'ogico}

El desarrollo del proyecto se puede dividir en dos partes principales, el análisis y la implementación de aplicación. \\

Para el análisis y la planificación se han utilizado las siguientes aplicaciones:
\begin{itemize}
	\item Oracle SQL Developer Data Modeler, una herramienta gráfica gratuita destinada a tareas de modelado de datos. Esta se ha empleado para realizar el diseño lógico y relacional de la base de datos.
	\item Magicdraw UML. Es una herramienta CASE aplicada al desarrollo de los diagramas UML, como los diagramas de casos de uso, de secuencias y los modelos de clases.
\end{itemize}

En cuanto a la implementación, las tecnologías empleadas han sido:
\begin{itemize}
	\item El lenguaje de programación PHP para el desarrollo de la aplicación web. Este se ha usado conjuntamente con:
		\begin{itemize}
			\item El lenguaje de marcado HTML.
			\item El lenguaje de diseño gráfico CSS.
			\item El lenguaje de programación Javascript.
		\end{itemize}
	\item El gestor de bases de datos MySQL.
	\item phpMyAdmin, para la administración de la base de datos a través del navegador.
	\item Sublime Text 3, como editor de texto del código.
	\item GitLab, una plataforma para el control de versiones del código.
\end{itemize}

Para la escritura de la memoria se ha utilizado LaTeX, a través del editor de documentos de texto TexMaker.
