% 1. Página en blanco
\thispagestyle{empty}\newpage~

% 2. Página en blanco
\newpage~\thispagestyle{empty}

% 3. Pagina con datos del TFG
\newpage~\thispagestyle{empty} 
\begin{center}
	{\fontsize{13.9}{14}\selectfont ESCUELA TÉCNICA SUPERIOR DE INGENIERÍA INFORMÁTICA GRADO EN INGENIERÍA INFORMÁTICA} \par \vspace{60pt}

	\textbf{{\fontsize{18}{18}\selectfont SISTEMA DE GESTIÓN DE TURNOS DE ACTUACIÓN PROFESIONAL DE UN COLEGIO DE INGENIERÍA}} \par \vspace{20pt}
	\textbf{{\fontsize{18}{18}\selectfont MANAGEMENT SYSTEM OF PROFESSIONAL ROLLS AT AN ENGINEERING PROFESSIONAL ASSOCIATION}} \par \vspace{60pt}

	{\fontsize{14}{14}\selectfont Realizado por} \\
	\textbf{{\fontsize{14}{14}\selectfont Borja Delgado Martín}} \par \vspace{20pt}
	{\fontsize{14}{14}\selectfont Tutorizado por} \\
	\textbf{{\fontsize{14}{14}\selectfont David Santo Orcero}} \par \vspace{20pt}
	{\fontsize{14}{14}\selectfont Departamento} \\
	\textbf{{\fontsize{14}{14}\selectfont Lenguajes y Ciencias de la Computación}} \par \vspace{20pt}

	{\fontsize{13}{13}\selectfont UNIVERSIDAD DE MÁLAGA} \\
	{\fontsize{13}{13}\selectfont MÁLAGA, SEPTIEMBRE DE 2019} \par \vspace{20pt}
\end{center}

\begin{flushright}
	{\fontsize{13}{13}\selectfont Fecha defensa: \_\_\_ de septiembre de 2019} \par \vspace{20pt}
\end{flushright}

{\fontsize{13}{13}\selectfont Fdo. El/la Secretario/a del Tribunal}


% 4. Página en blanco
\newpage~\thispagestyle{empty} 

% 5. Originalidad del trabajo
\newpage~\thispagestyle{empty} 
D/Dª.: \underline{\hspace{12cm}}, con DNI \underline{\hspace{4cm}}, estudiante del Grado en \underline{\hspace{7.2cm}} \underline{\hspace{9cm}}, de la Universidad de Málaga. \par \vspace{\baselineskip}

DECLARO QUE: \par \vspace{\baselineskip}

El Trabajo Fin de Grado denominado: \underline{\hspace{8.3cm}} \underline{\hspace{16cm}} \underline{\hspace{16cm}} es de mi autoría, inédito (no ha sido difundido por ningún medio, incluyendo internet) y original (no es copia ni adaptación de otra), no habiendo sido presentado anteriormente por mí ni por ningún otro autor ni en parte ni en su totalidad. Así mismo, se ha desarrollado respetando los derechos intelectuales de terceros, para lo cual se han indicado las citas necesarias en las páginas donde se usan, y sus fuentes originales se han incorporado en la bibliografía. Igualmente se han respetado los derechos de propiedad industrial o intelectual que pudiesen afectar a cualquier institución o empresa. \par \vspace{\baselineskip}

Para que así conste, firmo la presente declaración en Málaga, a \underline{\hspace{1cm}} de \\ \underline{\hspace{3cm}} de \underline{\hspace{2cm}}. \par \vspace{100pt}

Fdo.: D/Dª \underline{\hspace{10cm}}


%%Versión con margenes personalizados
%D/Dª.: \underline{\hspace{12cm}}, con DNI \underline{\hspace{4cm}}, estudiante del Grado en \underline{\hspace{5.2cm}} \underline{\hspace{9cm}}, de la Universidad de Málaga. \par \vspace{\baselineskip}

%DECLARO QUE: \par \vspace{\baselineskip}

%El Trabajo Fin de Grado denominado: \underline{\hspace{7.3cm}} \underline{\hspace{14.9cm}} \underline{\hspace{14.9cm}} es de mi autoría, inédito (no ha sido difundido por ningún medio, incluyendo internet) y original (no es copia ni adaptación de otra), no habiendo sido presentado anteriormente por mí ni por ningún otro autor ni en parte ni en su totalidad. Así mismo, se ha desarrollado respetando los derechos intelectuales de terceros, para lo cual se han indicado las citas necesarias en las páginas donde se usan, y sus fuentes originales se han incorporado en la bibliografía. Igualmente se han respetado los derechos de propiedad industrial o intelectual que pudiesen afectar a cualquier institución o empresa. \par \vspace{\baselineskip}

%Para que así conste, firmo la presente declaración en Málaga, a \underline{\hspace{1cm}} de \underline{\hspace{3cm}} de \underline{\hspace{2cm}}. \par \vspace{100pt}

%Fdo.: D/Dª \underline{\hspace{10cm}}


% 6. Página en blanco
\newpage~\thispagestyle{empty}

% 7. Resumen
\newpage~\thispagestyle{plain}
\section*{Resumen}

Los Turnos de Actuación Profesional (TAP) son un servicio destinado a atender las peticiones de trabajo profesional que se realicen por cualquier persona física y entidades públicas y privadas hacia los profesionales colegiados que voluntariamente figuren inscritos en el mismo. \par \vspace{\baselineskip}

Ante la complejidad derivada de gestionar un sistema de TAP, el Colegio Profesional de Ingeniería en Informática de Andalucía (CPIIA) se encontraba en la necesidad de implantar una aplicación que fuese capaz de organizar su gestión de la forma más automática posible, para, de esta forma, eliminar parte de la sobrecarga administrativa que estas suponen a las distintas comisiones del colegio. \par \vspace{\baselineskip}

El objetivo de este proyecto consiste en resolver esa problemática, desarrollando una aplicación web usando PHP y MySQL, además de otras tecnologías. \par \vspace{\baselineskip}

Para ello, se ha realizado una fase de análisis, que engloba la definición los requisitos que debe cumplir la aplicación, la estructura lógica que tendrá el sistema y la forma en que se cubrirán los requisitos antes mencionados. Todo esto cumpliendo con el reglamento genérico de turnos de actuación profesional del colegio. \par \vspace{\baselineskip}

Tras la implementación del sistema, se ha escrito un manual de usuario, donde se muestra la aplicación y la forma de usarla correctamente. \par \vspace{80pt}

\textbf{Palabras clave:} turnos de actuación profesional, TAP, colegio profesional, CPIIA, aplicación web, sistema de gestión, php, mysql. 


% 8. Abstract
\newpage~\thispagestyle{plain}
\section*{Abstract}

Professional Rolls (TAP) is a service designed to meet professional work requests made by any person and public and private entities towards collegiate professionals who voluntarily appear registered therein. \par \vspace{\baselineskip}

Due to the difficulty of managing a TAP system, the Professional Association of Computer Engineering of Andalusia (CPIIA) needed to get an application that could organize its management, automatizing as many tasks as possible. This way, it would reduce part of the administrative work overload that these involve to the different commissions of the association. \par \vspace{\baselineskip}

The aim of this project is to solve this problem, developing a web application using PHP and MySQL, in addition to other technologies. \par \vspace{\baselineskip}

For this, an analysis phase has been carried out, which includes the definition of the requirements that the application must fulfil, the logical structure that the system will have and the way in which the previous requirements will be met. All this, fulfilling what is required in the professional rolls generic regulation of the association. \par \vspace{\baselineskip}

After the implementation of the system, a user manual has been written, showing the application and how to use it correctly. \par \vspace{80pt}

\textbf{Keywords:} professional rolls, TAP, engineering professional association, CPIIA, web application, management system, php, mysql.

